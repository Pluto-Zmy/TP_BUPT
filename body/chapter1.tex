\pagenumbering{arabic}

\section{实验一——利用交换机简单组网}
	\subsection{实验设计}
		本实验希望仅通过二层交换来使三台计算机联网\\
		拓扑结构和ip如图\ref{fig:one-topology}
		三台主机都是同一个网段。 由于是独立网络, 所以我们没有用私有ip地址。
		\addfig[0.5]{one-topology.png}{fig:one-topology}{拓扑结构}
		此时交换机并没有设置任何vlan, 所以三个交换机该实验对交换机的端口没有要求.\\
		预测结果,三台主机都能够互相ping通.
	\subsection{实验结果}
		我们在三台windows主机上按照按照图\ref{fig:one-setip3}所示来配置IP.\\
		\addfig[0.5]{one-setip3.png}{fig:one-setip3}{设置主机三的ip}
		如图\ref{fig:one-3ping1}所示, 主机3能够ping通主机1, 如图\ref{fig:one-3ping2}所示, 主机3能够ping通主机2\\
		\addfig[0.5]{3ping1-good.png}{fig:one-3ping1}{主机3ping主机1}
		\addfig[0.5]{3ping2-good.png}{fig:one-3ping2}{主机3ping主机2}
	\subsection{实验结论}
		\begin{enumerate}
		  \item 独立网络不用私网地址也可以
		  \item 只用交换机搭建的二层网络网关无意义
          \item 交换机级联后, 可以视为同一个交换机, 主机能否ping同只取决于是否在同一个网段

		\end{enumerate}
\section{实验二——广播风暴}
	\subsection{实验设计}
		本实验希望验证两个交换机中接入两根网线产生的后果.
		拓扑结构如图\ref{fig:two-topology}\\
		\addfig[0.5]{two-topology.png}{fig:two-topology}{拓扑结构}
		预测结果, 接入两根线后会产生广播风暴, 导致无法ping通.\\
		如果在软件层面实现减断环路后就能够ping通.
		
	\subsection{实验结果}
		插入前换交换机黄等闪烁较慢(图\ref{fig:two-blink-single}), 两台主机能够相互ping通,
		插入后交换机黄灯闪烁变快(图\ref{fig:two-blink-double}), 两台主机不能够ping通(图\ref{fig:two-3ping1-bad})\\
		\addfig[0.5]{blink-single.jpg}{fig:two-blink-single}{只接一根线时的黄灯}
		\addfig[0.5]{blink-double.jpg}{fig:two-blink-double}{接两根线时的黄灯}
		\addfig[0.5]{3ping1-bad.png}{fig:two-3ping1-bad}{接两跟线时主机3无法ping通主机1}
		通过命令\emph{stp en}的方法剪断环路后, 主机就能ping通, 且黄灯恢复正常
		之后我们又尝试了在一台交换机上将一跟网线插到两个端口上, 结果一致

	\subsection{实验结论}
		\begin{enumerate}
		  \item 连接两根线后, 由于ARP协议的缺陷, 交换机会不断的发出询问, 由于成环, 该询问被自激, 所以产生广播风暴.
		  \item 从交换机的黄灯闪烁频率上能看出是否有广播风暴
		  \item 通过软件的方法可以避免广播风暴
		\end{enumerate}
\section{实验三--vlan}
	\subsection{实验设计}
		本实验希望验证vlan的隔离效果
		拓扑结构以及ip配置如图\ref{fig:three-topology}\\
		\addfig[0.5]{three-topology.png}{fig:three-topology}{拓扑结构以及ip配置}
	    预测结果,主机1和主机2能够相互ping通, 但是他们无法ping通主机3
	\subsection{实验结果}
		我们配置vlan如图\ref{fig:three-vlanconfig}
		\addfig[0.5]{three-vlanconfig.png}{fig:three-vlanconfig}{配置vlan}
		如图\ref{fig:three-1ping2-good}所示, 主机1能够ping通主机2。如图\ref{fig:three-1ping3-bad}所示, 主机1不能ping通主机3
		\addfig[0.5]{1ping2-good.png}{fig:three-1ping2-good}{主机1能够ping通主机2}
		\addfig[0.5]{1ping3-bad.png}{fig:three-1ping3-bad}{主机1不能够ping通主机3}
	\subsection{实验结论}
		\begin{enumerate}
		  \item vlan能够实现隔离,就算同一个网段,如果不是同一个vlan,也不能ping通
		\end{enumerate}
\section{实验四——利用路由器实现跨网段访问}
	\subsection{实验设计}
		本实验希望用两个路由器和一个交换机实现跨网段访问.\\
		拓扑结构以及ip配置如图\ref{fig:four-topology}\\
		\addfig[0.5]{four-topology.png}{fig:four-topology}{拓扑结构以及ip配置}
		其中交换机所涉及的端口都是同一个vlan之中.
		由于路由器的串口线不稳定, 所以我们采用的是路由器的\emph{subip}的方法, 并且采用静态路由实现路由器A B的连通.\\
		预测结果, 主机3够通过路由器A,B来ping通主机2.
	\subsection{试验结果}
	    我们按照图\ref{fig:four-setip3}的方式来配置ip
		\addfig[0.5]{four-setip3.png}{fig:four-setip3}{主机3配置ip}
		我们按照图\ref{fig:four-setrouteA}来配置路由器
		配置结果如图\ref{fig:four-routeconfigA}和图\ref{fig:four-routeconfigB}所示.\\
		\addfig[0.5]{four-setrouteA.png}{fig:four-setrouteA}{路由器A的配置方法}
		\addfig[0.5]{four-routeconfigA.png}{fig:four-routeconfigA}{路由器A的配置结果}
		\addfig[0.5]{four-routeconfigB.png}{fig:four-routeconfigB}{路由器B的配置结果}
		两个路由器添加静态路由表的结果如图\ref{fig:four-routetableA} 和\ref{fig:four-routetableB}所示
		\addfig[0.5]{four-routetableA.png}{fig:four-routetableA}{路由器A的静态路由}
		\addfig[0.5]{four-routetableB.png}{fig:four-routetableB}{路由器B的静态路由}
		主机3 tracert 主机2的结果如图\ref{fig:four-3tracert2}所示,可见主机3确实是通过路由器来跨网段连通到主机2
		\addfig[0.5]{four-3tracert2.png}{fig:four-3tracert2}{主机3tracert主机2}
	\subsection{实验结论}
		\begin{enumerate}
		  \item 路由器之间的连接除了串口还可以通过sub ip相连通
		\end{enumerate}
